\documentclass[swedish]{LnuCmThesis}

\usepackage{fixfoot}
\DeclareFixedFootnote{\fn:mainfiled}{Kurs i huvudmområdet}

\watermark[ARBETSUTGÅVA]

\documenttype{Utbildningsplan\\\Large Fakulteten för teknik}
\title{Webbprogrammerare}
\subtitle{120 högskolepoäng}
\renewcommand{\documentproperties}{%
    \emph{Ansvarig:} Johan Leitet\\
    \emph{Gäller från:} Höstterminen 2020\\
    \emph{Version:} 0.1\\
    \emph{Datum:} 2019-05-20\\
    \emph{Ämne:} Datavetenskap\\
    \emph{Nivå:} Grundnivå}

\begin{document}

\maketitle

\paragraph{Benämning svenska}

Webbprogrammerare, 120 högskolepoäng

\paragraph{Benämning engelska}

Web Development Programme, 120 credits

\paragraph{Nivå}

Grundnivå

\paragraph{Fastställande av utbildningsplan}

Fastställd 2009­09­15\\
Senast reviderad 2016­12­09 av fakultetsstyrelsen inom Fakulteten för teknik\\
Utbildningsplanen gäller från och med vårterminen 2017

\paragraph{Förkunskaper}

Grundläggande behörighet samt Matematik 2a / 2b / 2c eller Matematik B\\
(Områdesbehörighet 7/A7).

\section*{Programbeskrivning}

Utbildningen ska förbereda för yrkesroller inom områden där det krävs färdigheter i
programmering av webbapplikationer. Utbildningen är datavetenskaplig med inriktning
mot webbprogrammeringsområdet.

Fokus ligger på programmering och utveckling av robusta webbapplikationer med hög
kodkvalitèt för olika plattformar och med arbetssätt och metoder som är aktuella i
branschen.

Studenten utbildas för kompetens att utveckla nya webbaserade tjänster och produkter.
Efter utbildningen ska studenten vara väl förberedd för arbete på en IT­avdelning,
webbyrå eller IT-­företag med ansvar för utveckling av webbapplikationer med agila
utvecklingsmetoder. Studenten ska också se möjligheter i att starta eget företag.

\section*{Mål}

\subsection*{Högskoleförordningens examensordning: Examensmål}

\subsubsection*{Kunskap och förståelse}

För högskoleexamen skall studenten

\begin{itemize}
    \item visa kunskap och förståelse inom det huvudsakliga området (huvudområdet) för utbildningen, inbegripet kännedom om områdets vetenskapliga grund och kunskap om några tillämpliga metoder inom området.
\end{itemize}

\subsubsection*{Färdighet och förmåga}

För högskoleexamen skall studenten

\begin{itemize}
    \item visa förmåga att söka, samla och kritiskt tolka relevant information för att formulera svar på väldefinierade frågeställningar inom huvudområdet för utbildningen,
    \item visa förmåga att redogöra för och diskutera sitt kunnande med olika grupper, och
    \item visa sådan färdighet som fordras för att självständigt arbeta med vissa uppgifter inom det område som utbildningen avser.
\end{itemize}

\subsubsection*{Värderingsförmåga och förhållningssätt}

För högskoleexamen skall studenten
l visa kunskap om och ha förutsättningar för att hantera etiska frågeställningar
inom huvudområdet för utbildningen.

\subsection*{Programspecifika mål}

I ordningsföljd som speglar utbildningens progression, studenten skall kunna

\begin{itemize}
    \item skapa säkra, tillgängliga och användarvänliga webbapplikationer för olika typer av enheter,
    \item programmera robusta webbapplikationer med hög kodkvalitet,
    \item redogöra för webbens arkitektur, framväxt och utveckling samt betydelse/påverkan för dagens samhälle,
    \item leda och genomföra projekt inom webbprogrammering inkluderat kravställning, mjukvaruutveckling samt kontinuerlig leverans till driftsmiljö.
\end{itemize}

\section*{Innehåll och struktur}

\subsection*{Programöversikt}

Programmet har en programansvarig som tillser att utbildningen håller god kvalitet med
god progression samt tillser att kommunikationen mellan lärare och studenter fungerar
på ett bra sätt. Kvalitet och progression upprätthålls genom kontinuerlig dialog med
ämnesansvarig.

Årskurs ett i programmet fokuserar på att ge studenten en grundläggande kunskap om
klient­ och serverbaserade webbprogrammeringsmiljöer samt en grund i prototyp­- och
klassbaserad objektorienterad programmering. Vidare introduceras och tillämpas för
området relevanta programutvecklingsprocesser. Årskursen avslutas med en
sammanfattande projektkurs.

Årskurs två syftar till att ge studenten fördjupad kunskap om webbapplikationers
arkitektur och för webbappliaktioner relevanta programmeringsmönster samt kunskap
om och färdighet i objektorienterad analys och design. Testning av mjukvara och dess
kvalitet sätts i fokus tillsammans med förståelse för webben som en applikationsplattform. Under programmets sista termin genomförs ett kundorienterat
mjukvaruutvecklingsprojekt där en större webbapplikation skapas och driftsätts.
Årskursen avslutas med ett självständigt arbete.

\subsubsection*{Kurser i programmet}

Kurserna i programmet kan i samförstånd med programansvarig bytas ut mot
motsvarande kurser inom programmets inriktning. Vid utbyte av kurs kontrollerar
programansvarig att programmets mål fortfarande uppfylls. Förkunskapskraven för
kurser samt de lokala reglerna för examen vid Linnéuniversitetet måste alltid uppfyllas.

\paragraph*{Årskurs 1}

\begin{itemize}
    \item Grundläggande programmering 7,5 hp, Datavetenskap, G1N\fn:mainfiled{}. (Konstruktion av välstrukturerade program i Javascript. Prototypbaserad objektorienterad programmering.)
    \item Webbteknik 1 7,5 hp, Medieteknik, G1N. (Grundläggande webbdesign med klientbaserad teknik.)
    \item Grafiska verktyg 7,5 hp, Medieteknik, G1N. (Grundtekniker för skapande av vektorbaserad grafik, bitmapbaserad grafik samt 3D­grafik.)
    \item Klientbaserad webbprogrammering 7,5, G1F\fn:mainfiled{}. (Konstruktion av webbläsarbaserade applikationer med bland annat Javascript.)
    \item Serverbaserad webbprogrammering 7,5 hp, Datavetenskap, G1F\fn:mainfiled{}. (Konstruktion av webbserverbaserade applikationer samt förståelse för olika typer av webbservrar och dess kommunikation med webbklienten.)
    \item Programvaruteknik 7,5 hp, Datavetenskap, G1F\fn:mainfiled{}. (Introduktion i programvaruutveckling. Kursen fokuserarpå verktyg som kan användas under utveckling av programvara som stöd för bland annat modellering, konfigurationshantering och testning.)
    \item Objektorienterad programmering 7,5 hp, Datavetenskap, G1F\fn:mainfiled{}. (Klassbaserad objektorienterad programmering med C\#.)
    \item Individuellt mjukvaruutvecklingsprojekt 7,5 hp, Datavetenskap, G1F\fn:mainfiled{}. (Genomförande av ett mjukvaruprojekt där en fungerande mjukvara skall utvecklas med hjälp av de teoretiska och praktiska förutsättningar som getts i tidigare kurser.)
\end{itemize}

\paragraph*{Årskurs 2}

\begin{itemize}
    \item Objektorienterad analys och design med UML 7,5 hp, Datavetenskap, G1F\fn:mainfiled{}. (Kunskaper om hur objektorienterad analys och design kan implementeras i ett objektorienterat programmeringsspråk samt grunderna i modelleringsspråket UML.)
    \item Introduktion till mjukvarukvalitet 7,5 hp, Datavetenskap, G1F\fn:mainfiled{}. (Framställning av webbaserad mjukvara med hög kodkvalitet.)
    \item Mjukvarutestning 7,5 hp, Datavetenskap, G1F\fn:mainfiled{}. (Enhets­, system­ och integrationstestning, acceptanstestning.)
    \item Databasteori 7,5 hp, Datavetenskap, G1F\fn:mainfiled{}. (Metoder och teorier för databasdesign. Frågespråk, dokument­ och relationsdatabaser.)
    \item Arkitekturer och ramverk för webbapplikationer 7,5 hp, Datavetenskap, G1F\fn:mainfiled{}. (Förståelse för hur webbramverk och designmönster kan användas i samband med utveckling av webbapplikationer.)
    \item Webben som applikationsplattform 7,5 hp, Datavetenskap, G1F\fn:mainfiled{}. (Webbens arkitektur, framväxt och utveckling samt betydelse/påverkan för dagens samhälle.)
    \item Mjukvaruutvecklingsprojekt 7,5 hp, Datavetenskap, G1F\fn:mainfiled{}. (Genomförande av ett kundorienterat mjukvaruprojekt med ett fokus på projektledning och projektplanering.)
    \item Självständigt arbete 7,5 hp, Datavetenskap, G1E\fn:mainfiled{}. (Självständigt arbete omfattande teoretisk och experimentell verksamhet med åtföljande rapportskrivning och muntlig presentation.)
\end{itemize}

\subsubsection*{Arbetslivsanknytning}

Inom ramen för utbildningen har studenterna möjlighet att bedriva projekt tillsammans
med företag. Kurserna i utbildningen är i många fall utformade för att passa en framtida
anställning och stora delar av programmets innehåll är framtaget med feedback från
företag inom IT­sektorn.

\subsubsection*{Utlandsstudier}

Studenterna erbjuds ta del av det samlade utbudet av avtal med utländska lärosäten som
finns inom Linnéuniversitetet. Utlandsstudier sker i samråd med programansvarig och
under årskurs två.

\subsubsection*{Perspektiv i utbildningen}

Inom utbildningen arbetar man med hållbar utveckling genom att tillse att studenterna får
kunskaper som kommer att vara relevanta och applicerbara under en längre tidsperiod.
Hållbar utveckling syftar här på människan och dennes behov och hur man genom
tekniken kan tillse att dessa även i framtiden kan tillgodoses.

Internationaliseringsperspektivet inom utbildningen tillgodoses främst genom att ge
möjlighet för samverkan med universitet utomlands men även genom att hämta
inspiration och dra lärdomar av hur universitet internationellt bedriver undervisning och
forskning inom området.

\section*{Kvalitetsutveckling}

Kontinuerlig utvärdering och förbättring av programmet sker bland annat i samråd med
studenter i form av en programkommité, genom läsårsutvärderingar samt genom
samverkan med företag och andra intressenter samt genom benchmarking gentemot
andra högskolor och universitet.

Sammanställningar av programutvärderingar finns tillgängliga på programmets
webbplats.

Programansvarig ansvarar för att utvärderingen genomförs och att eventuella
kvalitetsproblem i programmet åtgärdas.

\section*{Examen}

Efter avklarade studier på programmet samt då avklarade studier motsvarar de
fordringar som finns angivna i Högskoleförordningens examensordning samt i den lokala
examensordningen för Linnéuniversitetet kan studenten ansöka om examen. De som
fullföljt programmet Webbprogrammerare 120 hp kan erhålla följande examen:

Högskoleexamen med inriktning mot webbprogrammering\\
Huvudområde: Datavetenskap

\textit{Higher Education Certificate in web development\\
Main field of study: Computer Science.}

Examensbeviset är tvåspråkigt (svenska/engelska). Tillsammans med examensbeviset
följer Diploma Supplement (engelska).

\section*{Övrigt}

Undervisningen bedrivs huvudsakligen på svenska, men inslag av engelska är
återkommande i form av till exempel engelskspråkig litteratur och lärresurser.

Campusstudier förutsätter egen tillgång till bärbar dator.

För att läsa utbildningen på distans så förutsätts egen tillgång till dator, headset,
webbkamera och internetuppkoppling.

Majoriteten av utbildningens lärresurser är publikt tillgängliga genom kursernas
webbplatser.

\end{document}