\documentclass[swedish]{LnuCmThesis}

\usepackage{fixfoot}
\DeclareFixedFootnote{\fn:mainfiled}{Kurs i huvudmområdet}

\watermark[ARBETSUTGÅVA]

\documenttype{Utbildningsplan\\\Large Fakulteten för teknik}
\title{Webbprogrammerare}
\subtitle{180 högskolepoäng}
\renewcommand{\documentproperties}{%
    \emph{Ansvarig:} Johan Leitet\\
    \emph{Gäller från:} Höstterminen 2020\\
    \emph{Version:} 0.1\\
    \emph{Datum:} 2019-05-20\\
    \emph{Ämne:} Datavetenskap\\
    \emph{Nivå:} Grundnivå}

\begin{document}

\maketitle

\paragraph{Benämning svenska}

Webbprogrammerare, 180 högskolepoäng

\paragraph{Benämning engelska}

Web Development Programme, 180 credits

\paragraph{Nivå}

Grundnivå

\paragraph{Fastställande av utbildningsplan}

Fastställd EJ FASTSTÄLLD\\
Utbildningsplanen gäller från och med höstterminen 2020

\paragraph{Förkunskaper}

Grundläggande behörighet samt Matematik 2a / 2b / 2c eller Matematik B\\
(Områdesbehörighet 7/A7).

\section*{Programbeskrivning}

Utbildningen ska förbereda för yrkesroller inom områden där det krävs färdigheter i programmering av webbapplikationer. Utbildningen är datavetenskaplig med inriktning mot webbprogrammeringsområdet.

Fokus ligger på programmering och utveckling av robusta webbapplikationer med hög kodkvalitet för olika plattformar och med arbetssätt och metoder som är aktuella i branschen. 

Studenten utbildas för kompetens att utveckla nya webbaserade tjänster och produkter. Efter utbildningen ska studenten vara väl förberedd för arbete på ett mjukvarutvecklingsföretag med ansvar för utveckling av webbapplikationer med agila utvecklingsmetoder samt driftsättning genom kontinuerlig leverans till driftsmiljö. Studenten ska också se möjligheter i att starta eget företag. 

\section*{Mål}

\subsection*{Högskoleförordningens examensordning: Examensmål}

\subsubsection*{Kunskap och förståelse}

För högskoleexamen skall studenten

\begin{itemize}
    \item visa kunskap och förståelse inom det huvudsakliga området (huvudområdet) för utbildningen, inbegripet kännedom om områdets vetenskapliga grund och kunskap om några tillämpliga metoder inom området.
\end{itemize}

\subsubsection*{Färdighet och förmåga}

För högskoleexamen skall studenten

\begin{itemize}
    \item visa förmåga att söka, samla och kritiskt tolka relevant information för att formulera svar på väldefinierade frågeställningar inom huvudområdet för utbildningen,
    \item visa förmåga att redogöra för och diskutera sitt kunnande med olika grupper, och
    \item visa sådan färdighet som fordras för att självständigt arbeta med vissa uppgifter inom det område som utbildningen avser.
\end{itemize}

\subsubsection*{Värderingsförmåga och förhållningssätt}

För högskoleexamen skall studenten
l visa kunskap om och ha förutsättningar för att hantera etiska frågeställningar
inom huvudområdet för utbildningen.

\subsection*{Programspecifika mål}

I ordningsföljd som speglar utbildningens progression, studenten skall kunna

\begin{itemize}
    \item versionshantera och skapa säkra, tillgängliga och användarvänliga webbapplikationer för olika typer av enheter,
    \item programmera, robusta webbapplikationer med hög kodkvalitet,
    \item redogöra för webbens arkitektur, framväxt och utveckling samt betydelse/påverkan för dagens samhälle
    \item \colorbox{green}{leda och} genomföra agila projekt inom webbprogrammering inkluderat kravställning och mjukvaruutveckling
    \item \colorbox{green}{skapa, anpassa och paketera webbapplikationer för automatiserad kontinuerlig leverans till driftsmiljö.}
\end{itemize}

\section*{Innehåll och struktur}

\subsection*{Programöversikt}
Programmet är treårigt men möjlighet ges att ansöka om en högskoleexamen med inriktning webbprogrammering inom datavetenskap efter två år. Detta förutsätter att kursen "Självständigt arbete" väljs under åk2.

Inom programmet ges möjlighet att läsa 30 hp i annat ämne varav 15 hp kan väljas bort till förmån för en fördjupning inom huvudområdet. Detta förutsätter dock att studenten tillskansat sig 15 hp i annat ämne än datavetenskap på annat sätt.

\textbf{Årskurs ett} i programmet fokuserar på att ge studenten en grundläggande kunskap om klient- och serverbaserade webbprogrammeringsmiljöer samt en grund i prototyp- och klassbaserad objektorienterad programmering. Vidare introduceras och tillämpas för området relevanta programutvecklingsprocesser. Årskursen avslutas med en sammanfattande projektkurs.

\textbf{Årskurs två} syftar till att ge studenten fördjupad kunskap om webbapplikationers arkitektur och för webbapplikationer relevanta programmeringsmönster samt kunskap om och färdighet i objektorienterad analys och design. Testning av mjukvara och dess kvalitet sätts i fokus tillsammans med förståelse för webben som en applikationsplattform. 
I slutet av årskursen har studenten möjlighet att välja att läsa ett självständigt arbete för att ansöka om högskoleexamen med inriktning mot Webbprogrammering (huvudområde datavetenskap)

\textbf{Årskurs tre} syftar till att ge studenten kunskaper om och färdigheter kring automatiserad kontinuerlig leverans av mjukvara i virtualiserade driftsmiljöer. Vidare ges fördjupade kunskaper kring algoritmer och datastrukturer.
Under programmets sista termin genomförs ett kundorienterat mjukvaruutvecklingsprojekt där en större webbapplikation skapas och driftsätts. 
Utbildningen avslutas med ett examensarbete.

\subsubsection*{Kurser i programmet}

Kurserna i programmet kan i samförstånd med programansvarig bytas ut mot
motsvarande kurser inom programmets inriktning. Vid utbyte av kurs kontrollerar
programansvarig att programmets mål fortfarande uppfylls. Förkunskapskraven för
kurser samt de lokala reglerna för examen vid Linnéuniversitetet måste alltid uppfyllas.

Studenten behöver läsa 15 hp kurser i annat ämne förutom de som ingår i programmet. Två kurser i ÅK2 går att byta ut mot valfri kurs i annat ämne.

\paragraph*{Årskurs 1}

\begin{itemize}
    \item Klientbaserad webbprogrammering 15 hp, Datavetenskap, G1N\fn:mainfiled{}. (Konstruktion av webbläsarbaserade applikationer med bland annat Javascript.)
    \item Webbteknik 1 7,5 hp, Medieteknik, G1N. (Grundläggande webbdesign med klientbaserad teknik.)
    \item Serverbaserad webbprogrammering 15 hp, Datavetenskap, G1F\fn:mainfiled{}. (Konstruktion av webbserverbaserade applikationer samt förståelse för olika typer av webbservrar och dess kommunikation med webbklienten.)
    \item Objektorienterad programmering 7,5 hp, Datavetenskap, G1F\fn:mainfiled{}. (Klassbaserad objektorienterad programmering.)
    \item Programvaruteknik 7,5 hp, Datavetenskap, G1F\fn:mainfiled{}. (Introduktion i programvaruutveckling. Kursen fokuserar på verktyg som kan användas under utveckling av programvara som stöd för bland annat modellering, versionshantering och testning.)
    \item Mjukvaruutvecklingsprojekt 7,5 hp, Datavetenskap, G1F\fn:mainfiled{}. (Genomförande av ett mjukvaruprojekt där en fungerande mjukvara skall utvecklas.)
\end{itemize}

\paragraph*{Årskurs 2}

\begin{itemize}
    \item Objektorienterad analys och design med UML 7,5 hp, Datavetenskap, G1F\fn:mainfiled{}. (Kunskaper om hur objektorienterad analys och design kan implementeras i ett objektorienterat programmeringsspråk samt grunderna i modelleringsspråket UML.)
    \item Mjukvarutestning 7,5 hp, Datavetenskap, G1F\fn:mainfiled{}. (Enhets­, system­ och integrationstestning, acceptanstestning.)
    \item Introduktion till mjukvarukvalitet 7,5 hp, Datavetenskap, G1F\fn:mainfiled{}. (Framställning av webbaserad mjukvara med hög kodkvalitet.)
    \item Datasäkerhet 7,5 hp, Datavetenskap, G1F\fn:mainfiled{}. (Introduktion till IT­säkerhet. Begrepp som risk, hot och säkerhetstjänster introduceras och exemplifieras. Fokus ligger på säkerheten i ett enskilt datorsystem)
    \item Databasteknik 7,5 hp, Datavetenskap, G1F\fn:mainfiled{}. (Metoder och teorier för databasdesign. Frågespråk, dokument­ och relationsdatabaser.)
    \item Webben som applikationsplattform 15 hp, Datavetenskap, G1F\fn:mainfiled{}. (Webbens arkitektur, framväxt och utveckling samt betydelse/påverkan för dagens samhälle.)
    \item Entreprenörskap och grundläggande affärsutveckling 7,5 hp, Företagsekonomi, G1N. (Entreprenörsskap och grundläggande affärsutveckling.)
    \item Självständigt arbete 7,5 hp, Datavetenskap, G1E\fn:mainfiled{}. (Självständigt arbete omfattande teoretisk och experimentell verksamhet med åtföljande rapportskrivning och muntlig presentation.)
    * Valfri(a) kurs(er) i annat ämne
\end{itemize}

\paragraph*{Årskurs 3}

\begin{itemize}
    \item Algoritmer och datastrukturer 7,5 hp, Datavetenskap, G1F\fn:mainfiled{}. (Fördjupade studier av datastrukturer, samt algoritmval och effektivitet.)
    \item Web Intelligence 7,5 hp, Datavetenskap, G2F\fn:mainfiled{}. (Teoretiska och praktiska kunskaper inom maskininlärning, data utvinning och bearbetning, sökmotorer och rekommendationssystem.)
    \item Cloud-native applications 15 hp, Datavetenskap, G2F\fn:mainfiled{}. (Automatiserad kontinuerlig leverans av mjukvara i virtualiserade driftsmiljöer.)
    \item Mjukvaruutvecklingsprojekt 15 hp, Datavetenskap, G2F\fn:mainfiled{}. (Genomförande av ett kundorienterat mjukvaruprojekt med ett fokus på projektledning och projektplanering.)
    \item Examensarbete 15 hp, Datavetenskap, G2F\fn:mainfiled{}.
\end{itemize}

\subsubsection*{Samhällsrelevans}
Programmets studenter får vid flera tillfällen under programmets gång möta representanter från arbetslivet. Flera kurser inbjuder gästföreläsare och har delat kursvärdskap där företag eller organisationer stöttar kursledningen i frågor som rör kursens innehåll samt relevans i förhållande till det omgivande samhället. I flera kurser genomförs projekt som helt eller delvis kan förläggas hos avnämare.

Kurserna i utbildningen är i många fall utformade för att passa en framtida anställning och stora delar av programmets innehåll är framtaget med återkoppling från företag inom mjukvaruutvecklingssektorn.

\subsubsection*{Internationalisering}
Under främst tredje året kan utlandsstudier bedrivas under en eller två terminer. Kursurval görs i samråd med programansvarig för att underlätta ett kommande tillgodoräknande inom utbildningsprogrammet.

\subsubsection*{Perspektiv i utbildningen}

\textit{Hållbar utveckling}
Distans, inspelningar
Genom att erbjuda utbildningen på distans så möjliggörs ett livslångt lärande och möjlighet till fortbildning oavsett livssituation.

\textit{Lika villkor}
Centralt för utbildningen är att skapa förutsättningar för alla människors lärande och utveckling. 

Oavsett om utbildningen läses på distans eller på plats så ges den studerande samma eller motsvarande undervisning och examination.

\textit{Breddat kunskapsperspektiv}
Kurs i annat ämne

\textit{Entreprenöriellt förhållningssätt}
För att understödja “Det entreprenöriella universitetet” så inkluderas en kurs i affärss…. vidare så samarbetar kursledningen med nätverk och organisationer som stödjer studenterna i processer som rör egenföretagande.
I mötet med företag i…...

\section*{Kvalitetsutveckling}

Kontinuerlig utvärdering och förbättring av programmet sker bland annat i samråd med
studenter i form av en programkommité, genom läsårsutvärderingar samt genom
samverkan med företag och andra intressenter samt genom benchmarking gentemot
andra högskolor och universitet.

Sammanställningar av programutvärderingar finns tillgängliga på programmets
webbplats.

Programansvarig ansvarar för att utvärderingen genomförs och att eventuella
kvalitetsproblem i programmet åtgärdas.

\section*{Examen}

Efter avklarade studier på programmet samt då avklarade studier motsvarar de
fordringar som finns angivna i Högskoleförordningens examensordning samt i den lokala
examensordningen för Linnéuniversitetet kan studenten ansöka om examen. De som
fullföljt programmet Webbprogrammerare 180 hp kan erhålla följande examen:

Filosofie kandidatexamen med inriktning mot webbprogrammering\\
Huvudområde: Datavetenskap

\textit{Bachelor if Science in web development\\
Main field of study: Computer Science.}

Examensbeviset är tvåspråkigt (svenska/engelska). Tillsammans med
examensbeviset följer Diploma Supplement (engelska).

\section*{Övrigt}
Undevisningen bedrivs på helfart, dagtid.

Undervisningen bedrivs huvudsakligen på svenska, men inslag av engelska är
återkommande i form av till exempel engelskspråkig litteratur och lärresurser.

Campusstudier förutsätter egen tillgång till bärbar dator.

För att läsa utbildningen på distans så förutsätts egen tillgång till dator, headset,
webbkamera och internetuppkoppling utav god kvalitet.

Majoriteten av utbildningens lärresurser är publikt tillgängliga genom kursernas
webbplatser.

\end{document}