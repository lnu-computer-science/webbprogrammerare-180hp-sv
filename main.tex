\documentclass[swedish]{LnuCmThesis}

\usepackage{fixfoot}
\DeclareFixedFootnote{\fn:mainfield}{Kurs i huvudmområdet}
\watermark[ARBETSUTGÅVA]

\documenttype{Inrättande av nytt program\\\Large Fakulteten för teknik}
\title{Webbprogrammerare}
\subtitle{180 högskolepoäng}
\renewcommand{\documentproperties}{%
    \emph{Ansvarig:} Johan Leitet\\
    \emph{Gäller från:} Höstterminen 2020\\
    \emph{Version:} 0.3\\
    \emph{Datum:} \today\\
    \emph{Ämne:} Datavetenskap\\
    \emph{Nivå:} Grundnivå}

\begin{document}

\maketitle
\section*{Bakgrund}
\textit{Webbprogrammerare 120 hp (WP)} har under lång tid varit en av de populäraste utbildningarna vid universitetet. Även nationellt är den en av de mest sökta IT-utbildningarna. Utbildningen har ett starkt fokus mot konstruktion av webbapplikationer. Utbildning bedrivs idag parallellt med utbildningen \textit{Utveckling och drift av mjukvarusystem 180 hp (UDM)} och många kurser i de två utbildningarna samläses. Den senare har dels fokus mot konstruktion av mjukvara men även mot drift av densamma. Att idag ha ett program med fokus mot drift och en med fokus mot utveckling är olyckligt då trenderna i branschen pekar mot att låta utveckling och drift gå hand i hand (DevOps). Detta behov identifierades i och med införandet av UDM 2015 men även WP behöver göra förändringar för att möta detta växande behov. Detta skulle dock göra UDM och WP allt för lika och utan någon unik profil. Av denna anledning avses att kraftsamla kring den mest inarbetade utbildningen, Webbprogrammerare. Förhoppningen är att hålla kvar det goda söktryck som dessa utbildningar har idag. Totalt antal förstahandssökande HT2019 (520) motsvarar cirka 25\% av fakultetens totalt antal förstahandssökande.

\paragraph{Önskemål}
Önskemålet är att införa Webbprogrammerare 180 hp samtidigt som Utveckling och drift av mjukvarusystem avvecklas. Webbprogrammerare 120 hp finns kvar i utbudet och utbildningsplanerna för WP120/180 är konstruerade på sådant sätt att oavsett vilken ingång studenten sökt till så går det att avsluta utbildningen efter två (högskoleexamen) alternativt tre år (kandidatexamen). \textit{Det senare kräver dock att studenten söker och antas till senare del av program (WP180).}

\paragraph{Former och dimentionering}
Med gjorda ändringar skulle följande utbildningsutbud kunna erbjudas med start HT2020:
\begin{itemize}
    \item Webbprogrammerare 120 hp, Platsoberoende* (Planeringstal, Åk1 70 studenter)
    \item Webbprogrammerare 180 hp, Kalmar (Planeringstal, Åk1 40 studenter)
    \item Webbprogrammerare 180 hp, Distans (Planeringstal Åk1 50 studenter)
\end{itemize}
\textit{* En ansökninskod, studenten väljer själv form, campus Kalmar eller distans}

Dimentioneringen motsvarar dagens Webbprogrammerarutbildning sammantaget med dagens Utveckling och drift av mjukvarusystem. Samtliga utbildningar bedrivs på helfart, dagtid.

\paragraph{Förhållande till övrigt utbildningsutbud}
Utbildningens kurser är samordnade med kommande ändringar i andra datavetenskapliga program till följd av införandet av civilingenjörsutbildingen.

\clearpage

\section*{Utbildningsplan, Webbprogrammerare 180 hp}
\paragraph{Benämning svenska}

Webbprogrammerare, 180 högskolepoäng

\paragraph{Benämning engelska}

Web Development Programme, 180 credits

\paragraph{Nivå}

Grundnivå

\paragraph{Fastställande av utbildningsplan}

Fastställd EJ FASTSTÄLLD\\
Utbildningsplanen gäller från och med höstterminen 2020

\paragraph{Förkunskaper}

Grundläggande behörighet samt Matematik 2a / 2b / 2c eller Matematik B\\
(Områdesbehörighet 7/A7).

\section*{Programbeskrivning}

Utbildningen ska förbereda för yrkesroller inom områden där det krävs färdigheter i programmering av webbapplikationer. Utbildningen ger en bred datavetenskaplig grund med en inriktning mot webbprogrammeringsområdet.

Fokus ligger på programmering, utveckling och driftsättning av robusta webbapplikationer med hög kodkvalitet för olika plattformar och med arbetssätt och metoder som är aktuella i branschen. 

Studenten utbildas för kompetens att utveckla nya webbaserade tjänster och produkter. Efter utbildningen ska studenten vara väl förberedd för arbete på ett mjukvarutvecklingsföretag med ansvar för utveckling av webbapplikationer med agila utvecklingsmetoder samt driftsättning genom kontinuerlig leverans till driftsmiljö. Studenten ska också se möjligheter i att starta eget företag under eller efter utbildningen. 

\section*{Mål}

\subsection*{Högskoleförordningens examensordning: Examensmål}

\subsubsection*{Kunskap och förståelse}

För kandidatexamen skall studenten

\begin{itemize}
    \item visa kunskap och förståelse inom huvudområdet för utbildningen, inbegripet kunskap om områdets vetenskapliga grund, kunskap om tillämpliga metoder inom området, fördjupning inom någon del av området samt orientering om aktuella forskningsfrågor.
\end{itemize}

\subsubsection*{Färdighet och förmåga}

För kandidatexamen skall studenten

\begin{itemize}
    \item visa förmåga att söka, samla, värdera och kritiskt tolka relevant information i en problemställning samt att kritiskt diskutera företeelser, frågeställningar och situationer,
    \item visa förmåga att självständigt identifiera, formulera och lösa problem samt att genomföra uppgifter inom givna tidsramar,
    \item visa förmåga att muntligt och skriftligt redogöra för och diskutera information, problem och lösningar i dialog med olika grupper, och
    \item visa sådan färdighet som fordras för att självständigt arbeta inom det område som utbildningen avser.
\end{itemize}

\subsubsection*{Värderingsförmåga och förhållningssätt}

För kandidatexamen skall studenten
\begin{itemize}
    \item visa förmåga att inom huvudområdet för utbildningen göra bedömningar med hänsyn till relevanta vetenskapliga, samhälleliga och etiska aspekter,
    \item visa insikt om kunskapens roll i samhället och om människors ansvar för hur den används,
    och
    \item visa förmåga att identifiera sitt behov av ytterligare kunskap och att utveckla sin kompetens.
\end{itemize}

\subsection*{Programspecifika mål}

I ordningsföljd som speglar utbildningens progression, studenten skall kunna

\begin{itemize}
    \item versionshantera och skapa säkra, tillgängliga och användarvänliga webbapplikationer för olika typer av enheter,
    \item programmera, robusta webbapplikationer av hög kodkvalitet,
    \item redogöra för webbens arkitektur, framväxt och utveckling samt betydelse/påverkan för dagens samhälle
    \item leda och genomföra agila projekt inom webbprogrammering inkluderat kravställning och mjukvaruutveckling
    \item skapa, anpassa och paketera webbapplikationer för automatiserad kontinuerlig leverans till driftsmiljö.
\end{itemize}

\section*{Innehåll och struktur}

\subsection*{Programöversikt}
Programmet är treårigt men möjlighet ges att ansöka om en högskoleexamen med inriktning webbprogrammering (huvudområde datavetenskap) efter två år. Detta förutsätter att kursen ''Självständigt arbete'' väljs under åk2.

Inom programmet ges möjlighet att läsa 30 hp i annat ämne varav 15 hp kan väljas bort till förmån för en fördjupning inom huvudområdet. Detta förutsätter dock att studenten tillskansat sig 15 hp i annat ämne än datavetenskap på annat sätt.

\textbf{Årskurs ett} i programmet fokuserar på att ge studenten en grundläggande kunskap om klient- och serverbaserade webbprogrammeringsmiljöer samt en grund i prototyp- och klassbaserad objektorienterad programmering. Vidare introduceras och tillämpas för området relevanta programutvecklingsprocesser. Årskursen avslutas med en sammanfattande projektkurs.

\textbf{Årskurs två} syftar till att ge studenten fördjupad kunskap om webbapplikationers arkitektur och för webbapplikationer relevanta designmönster samt kunskap om och färdighet i objektorienterad analys och design. Webben som en applikationsplattform sätts i fokus och testning av mjukvara samt dess kvalitet är centralt. 
I slutet av årskursen har studenten möjlighet att välja att läsa ett självständigt arbete för att ansöka om högskoleexamen med inriktning mot Webbprogrammering (huvudområde datavetenskap)

\textbf{Årskurs tre} syftar till att ge studenten kunskaper om och färdigheter kring automatiserad kontinuerlig leverans av mjukvara i virtualiserade driftsmiljöer. Vidare ges fördjupade kunskaper kring algoritmer och datastrukturer.
Under programmets sista termin genomförs ett kundorienterat mjukvaruutvecklingsprojekt där en större webbapplikation skapas och driftsätts. 
Utbildningen avslutas med ett examensarbete.

\subsubsection*{Kurser i programmet}

Kurserna i programmet kan i samförstånd med programansvarig bytas ut mot
motsvarande kurser inom programmets inriktning. Vid utbyte av kurs kontrollerar
programansvarig att programmets mål fortfarande uppfylls. Förkunskapskraven för
kurser samt de lokala reglerna för examen vid Linnéuniversitetet måste alltid uppfyllas.

Studenten behöver läsa 15 hp kurser i annat ämne förutom de som ingår i programmet. Två kurser i ÅK2 går att byta ut mot valfri kurs i annat ämne.

\paragraph*{Årskurs 1}

\begin{itemize}
    \item Klientbaserad webbprogrammering 15 hp, Datavetenskap, G1N\fn:mainfield{}. (Konstruktion av webbläsarbaserade applikationer med bland annat Javascript.)
    \item Webbteknik 1 7,5 hp, Medieteknik, G1N. (Grundläggande webbdesign med klientbaserad teknik.)
    \item Serverbaserad webbprogrammering 15 hp, Datavetenskap, G1F\fn:mainfield{}. (Konstruktion av webbserverbaserade applikationer samt förståelse för olika typer av webbservrar och dess kommunikation med webbklienten.)
    \item Objektorienterad programmering 7,5 hp, Datavetenskap, G1F\fn:mainfield{}. (Klassbaserad objektorienterad programmering.)
    \item Programvaruteknik 7,5 hp, Datavetenskap, G1F\fn:mainfield{}. (Introduktion i programvaruutveckling. Kursen fokuserar på verktyg som kan användas under utveckling av programvara som stöd för bland annat modellering, versionshantering och testning.)
    \item Mjukvaruutvecklingsprojekt 7,5 hp, Datavetenskap, G1F\fn:mainfield{}. (Genomförande av ett mjukvaruprojekt där en fungerande mjukvara skall utvecklas.)
\end{itemize}

\paragraph*{Årskurs 2}

\begin{itemize}
    \item Objektorienterad analys och design med UML 7,5 hp, Datavetenskap, G1F\fn:mainfield{}. (Kunskaper om hur objektorienterad analys och design kan implementeras i ett objektorienterat programmeringsspråk samt grunderna i modelleringsspråket UML.)
    \item Mjukvarutestning 7,5 hp, Datavetenskap, G1F\fn:mainfield{}. (Enhets­, system­ och integrationstestning, acceptanstestning.)
    \item Introduktion till mjukvarukvalitet 7,5 hp, Datavetenskap, G1F\fn:mainfield{}. (Framställning av webbaserad mjukvara med hög kodkvalitet.)
    \item Datasäkerhet 7,5 hp, Datavetenskap, G1F\fn:mainfield{}. (Introduktion till IT­säkerhet. Begrepp som risk, hot och säkerhetstjänster introduceras och exemplifieras. Fokus ligger på säkerheten i ett enskilt datorsystem)
    \item Databasteknik 7,5 hp, Datavetenskap, G1F\fn:mainfield{}. (Metoder och teorier för databasdesign. Frågespråk, dokument­ och relationsdatabaser.)
    \item Webben som applikationsplattform 15 hp, Datavetenskap, G1F\fn:mainfield{}. (Webbens arkitektur, framväxt och utveckling samt betydelse/påverkan för dagens samhälle.)
    \item Entreprenörskap och grundläggande affärsutveckling 7,5 hp, Företagsekonomi, G1N. (Entreprenörsskap och grundläggande affärsutveckling.)
    \item Självständigt arbete 7,5 hp, Datavetenskap, G1E\fn:mainfield{}. (Självständigt arbete omfattande teoretisk och experimentell verksamhet med åtföljande rapportskrivning och muntlig presentation.)
    \item Valfri(a) kurs(er) i annat ämne
\end{itemize}

\paragraph*{Årskurs 3}

\begin{itemize}
    \item Algoritmer och datastrukturer 7,5 hp, Datavetenskap, G1F\fn:mainfield{}. (Fördjupade studier av datastrukturer, samt algoritmval och effektivitet.)
    \item Web Intelligence 7,5 hp, Datavetenskap, G2F\fn:mainfield{}. (Teoretiska och praktiska kunskaper inom maskininlärning, data utvinning och bearbetning, sökmotorer och rekommendationssystem.)
    \item Cloud-native applications 15 hp, Datavetenskap, G2F\fn:mainfield{}. (Automatiserad kontinuerlig leverans av mjukvara i virtualiserade driftsmiljöer.)
    \item Mjukvaruutvecklingsprojekt 15 hp, Datavetenskap, G2F\fn:mainfield{}. (Genomförande av ett kundorienterat mjukvaruprojekt med ett fokus på projektledning och projektplanering.)
    \item Examensarbete 15 hp, Datavetenskap, G2F\fn:mainfield{}.
\end{itemize}

\subsubsection*{Samhällsrelevans}
Programmets studenter får vid flera tillfällen under programmets gång möta representanter från arbetslivet. Flera kurser inbjuder gästföreläsare och har delat kursvärdskap där företag eller organisationer stöttar kursledningen i frågor som rör kursens innehåll samt relevans i förhållande till det omgivande samhället. I flera kurser genomförs projekt som helt eller delvis kan förläggas hos avnämare.

Kurserna i utbildningen är i många fall utformade för att passa en framtida anställning och stora delar av programmets innehåll är framtaget med återkoppling från företag inom mjukvaruutvecklingssektorn.

\subsubsection*{Internationalisering}
Under främst tredje året kan utlandsstudier bedrivas under en eller två terminer. Kursurval görs i samråd med programansvarig för att underlätta ett kommande tillgodoräknande inom utbildningsprogrammet.

\subsubsection*{Perspektiv i utbildningen}

\textit{Hållbar utveckling}
Genom att erbjuda utbildningen på distans så möjliggörs ett livslångt lärande och möjlighet till fortbildning oavsett livssituation. Många av utbildningens lärresurser är dessutom öppna för allmänheten att ta del av och för andra lärosäten att återanvända.

\textit{Lika villkor}
Centralt för utbildningen är att skapa förutsättningar för alla människors lärande och utveckling. Utbildningen bejakar individens olika lärstilar genom att erbjuda och träna studenten i olika typer av lär- och examinationsformer. 

Oavsett om utbildningen läses på distans eller på plats så ges den studerande samma eller motsvarande undervisning och examination.

\textit{Breddat kunskapsperspektiv}
Inom utbildningen erbjuds studenten kurser i ämnen utanför huvudområdet med syftet att bredda studentens bildande. 
Från årskurs två erbjuds studenten att själv välja utvecklingsmiljö och programmeringsspråk för att uppmuntra och träna studenten i att ta ansvar för sin egen kunskapsutveckling.

\textit{Entreprenöriellt förhållningssätt}
“Det entreprenöriella Linnéuniversitetet” syftar till att skapa förutsättningar för ett brett entreprenörsskap. Genom flertalet projektkurser tillsammans med en kurs i entreprenörsskap och grundläggande affärsutveckling rustas våra studenter att medvetet och kritiskt bidra till att ge utrymme för socialt, ekonomiskt och ekologiskt hållbart entreprenörskap.
Vidare så samarbetar kursledningen med nätverk och organisationer som stödjer studenterna i processer som rör egenföretagande.

\section*{Kvalitetsutveckling}

Kontinuerlig utvärdering och förbättring av programmet sker bland annat i samråd med studenter i form av en programkommité, genom läsårsutvärderingar samt genom samverkan med företag och andra intressenter samt genom benchmarking gentemot andra högskolor och universitet.

Sammanställningar av programutvärderingar finns tillgängliga på programmets webbplats.

Programansvarig ansvarar för att utvärderingen genomförs och att eventuella kvalitetsproblem i programmet åtgärdas.

\section*{Examen}

Efter avklarade studier på programmet samt då avklarade studier motsvarar de fordringar som finns angivna i Högskoleförordningens examensordning samt i den lokala examensordningen för Linnéuniversitetet kan studenten ansöka om examen. De som fullföljt programmet Webbprogrammerare 180 hp kan erhålla följande examen:

Filosofie kandidatexamen med inriktning mot webbprogrammering\\
Huvudområde: Datavetenskap

\textit{Bachelor of Science in web development\\
Main field of study: Computer Science.}

Examensbeviset är tvåspråkigt (svenska/engelska). Tillsammans med
examensbeviset följer Diploma Supplement (engelska).

\section*{Övrigt}
Undevisningen bedrivs på helfart, dagtid.

Undervisningen bedrivs huvudsakligen på svenska, men inslag av engelska är
återkommande i form av till exempel engelskspråkig litteratur och lärresurser.

Campusstudier förutsätter egen tillgång till bärbar dator.

För att läsa utbildningen på distans så förutsätts egen tillgång till dator, headset,
webbkamera och internetuppkoppling för att kunna ta del av högupplösa, strömmade lärresurser.

Majoriteten av utbildningens lärresurser är publikt tillgängliga genom kursernas
webbplatser.

\end{document}